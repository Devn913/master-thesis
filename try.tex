\documentclass[12pt,a4paper,oneside,english]{report}
\begin{document}
% Given a graph \(G = (V,E), \ V = \{v_1,v_2,...,v_n\} \text{ and } E = \{e_1,e_2,...,e_n\} \) are the vertex and hyper-edges set.

% We can define a vertex-edge matrix \(F\) such that
% \begin{equation*} \label{eq:13}
  \[ \delta(e) = \sum_{v \in e} f(v,e);\quad  f(v,e) = \left\{
\begin{array}{ll}
      1,\ v \in e\\
      0,\  v \notin e
\end{array} 
\right. \]
% \end{equation*}
% The degree of hyperedge \( \delta(e) \) is defined as the number of vertices a hyperedge contains and $d(v)$ represents degree of a vertex,
\begin{eqnarray*}  d(v) = \sum_{v \in e, e \in E} w(e) = \sum_{e \in E} w(e)f(v,e) \quad w(e) = \frac{1}{\delta(e)(\delta(e)-1)} \sum_{\{v_i,v_j\} \in e} exp \left( -\frac{||v_i - v_j||^2}{\mu}\right)
\quad\end{eqnarray*}
% The following formula defines the hyperedge weight $w(e)$:


% Similar to a simple graph, the Hypergraph's Laplacian matrix can be defined as \cite{Zhou}:
% \begin{equation*} \label{eq:14}
  \[  H_s = D_v - FD_wD_e^{-1}F^T\]
% \end{equation*}
% where \(D_v, D_e, D_w\) are the diagonal matrices composed of \(d(v),\delta(e) \\ \text{ and } w(e) \) respectively. 
% \\ 
% According to Zhou's \cite{Zhou}, the Laplacian regularized Hypergraph matrix can also be defined as 
% \begin{equation}\label{eq:15}
%     H_s = I - D_v^{-\frac{1}{2}}FD_wD_e^{-1}F^TD_v^{-\frac{1}{2}}
% \end{equation}

% Matrix $H_s$ is a symmetric and positive definite matrix.
\end{document}